\documentclass[12pt,letterpaper]{article}
\bibliographystyle{plain}
\bibliography{BIBTeX}
\usepackage{amsmath}
\usepackage{amsthm}
\usepackage{amssymb}
\usepackage{amsfonts}
\usepackage{pdfsync}
\usepackage{caption}
\usepackage{color}
\usepackage{bm}

\usepackage{graphicx}


\theoremstyle{definition}
\newtheorem{dfn}{Definition}

\begin{document}

% The numbers below controls the amount of space between the following sections
\def\shiftdowna{0.32in}  % Adjust for balance
\def\shiftdownb{0.22in}  % Adjust for balance

% Set up the boiler plate at the top of the page

\begin{center}
\textbf{{\large Project Peer Evaluation}}\\


% TITLE
\vspace \shiftdowna
\textbf{{\large Constructing Dedicated Porfolio against District Bond Obligations from a Simplified Scenario}}

\vspace\shiftdownb
\underline{Sponsor}\\
\vspace{5pt}
\text{Stone \& Youngberg}

% STUDENTS
\vspace \shiftdownb
\underline {Participants} \\
\vspace{5pt}
\text{Zhenhan Zhao}, \texttt{zzhao13@jhu.edu}\\
\text{Shihong Li}, \texttt{sli50@jhu.edu}\\
% SPONSORS
\vspace \shiftdownb
\underline {Project Mentor}\\
\vspace{5pt}
\text{Nam Lee}, \texttt{nhlee@jhu.edu}

% DATE
\vspace{.50in}
\vspace \shiftdowna
Date: \today

\end{center}


\newpage

\section{Zhenhan's Evaluation} 
 {\bf My strength.} I have a strong confidence in our project's real-world application. Usually after talking with Dr.Lee about the project, I'd constantly seek any improvement opportunities in my project. As we decided to do my individual project as the group one, I have effectively and efficiently communicated with Shihong and told her every thought of mine. For each task, I paid strong attention to details. In meeting deadlines, I budgeted time effectively. I responded quickly to any changes in the project or the schedule. When the changes became very stressful, I could still keep my passion and enthusiasm on the project. \\
\vspace{1 mm}
\noindent  \\
 {\bf My weekness.} I need to work accordingly to the specific instructions of each task, sometimes I might leave out several tiny instructions. Insomnia isn't an excuse for me not being able to finish the journals on time. I might need to also pay attention to make our project more presentable and complete. Being unfamiliar with LaTex, I need to work faster on this software to save more time. Even have met the deadline, I still could work earlier with some tasks to avoid procrastination. I usually had too much doubt on new information, in the future, I might need to improve my ability to process and judge the new information. Too much confidence in me sometimes clouded my attention to Shihong's opinions.\\
\vspace{1 mm}
\noindent  \\
{\bf My performance point: 8.5/10}\\
\vspace{5 mm}
\noindent  \\
 {\bf Shihong's strength.} After we decided to do my individual project, Shihong had adjusted to the new one really quick. She literally process every word of my work statement and even came up with some great new ideas. She always has a very positive perspective. Even though a lot of times I had to share some different idea, she would listen to me and give advice to me. She's a great person to work with. Every time we had pressure, she'd tell jokes and try to talk me into facing those pressures. She contributed a lot to our final report. Because, we are not allowed to use other software but R, she took the major responsibility of building our R package. \\
\vspace{1 mm}
\noindent  \\
 {\bf Shihong's weekness.} One problem of our team is we didn't pick whom to write the team journals. Throughout the whole project, I completely forgot this journal thing, Shihong should have reminded me or just written the journals directly.  Sometimes she finished the task really fast but without the equivalent quality. However, some of the tasks needed further editing and correction. She might need to work on her ability to speak out her opinions. There were several times I knew that she had something to say but she just hesitated and those ideas in her mind could be a major breakthrough to our project. The other thing she needs to improve or we both need to is to act fast instead of postponing the task till the last minute before deadline.\\
\vspace{1 mm}
\noindent  \\
{\bf Shihong Performance point: 9/10}\\

\section{Shihong's Evaluation}
 {\bf My strength.} I digested the new project's information very fast. For each task, I took on strong responsibility to finish each one. Even without specific guidance, I was able to work well with R package building. I displayed effective communication skills. Under the pressure, I would tell jokes to ease the tension within my team. I listened carefully to Zhenhan's opinion and has a good agreeable personality. Throughout the project,  I kept my enthusiasm towards the project regardless of how challenging it it. \\
\vspace{1 mm}
\noindent  \\
 {\bf My weekness.} I should've taken the responsibility of writing the journals or helping Zhenhan finishing the journals. Sometimes I lost efficiency by working too fast. This is probably why, we missed some major terminology explanation at our final video and progress report. In retrospect to our project, I've found that the level of my effort in tasks is inconsistent. Obviously, I paid too much effort to the written materials rather than the videos. I need to work better and more efficiently without supervision. Most of my tasks were finished just before the deadlines, this is still an example of procrastination, a very serious problem in the team too. \\
\vspace{1 mm}
\noindent  \\
{\bf My performance point: 9/10}\\
\vspace{5 mm}
\noindent  \\
 {\bf Zhenhan's strength.} Zhenhan has always  been very confident in our project. To fill me in quickly with the project, he explained very thoroughly and with great details. He has displayed a very impressive ability of effective communication. Every time after he talked with Dr. Lee, he would propose some new ideas and directions toward our project. All these ideas and directions were described very clear and exact by him. He has a strong mastery of time management. He always led the team to survive before any deadline. \\
\vspace{1 mm}
\noindent  \\
{\bf Zhenhan's weekness.}  He should have followed the instructions more carefully. There were several times we missed a few requirements and had some scores deducted. Even though the team journals are not part of the grading, he still should've paid more attention to these forms. Sometimes he focuses too much on his ideas and may neglect the possibility of my ideas. In terms of the R, he might need to practice more on R besides other software. Some of the slides he made contained too many words or too concise. He also, could do better at the layout of our presentation slides. \\
\vspace{1 mm}
\noindent  \\
{\bf Zhenhan's performance point: 8/10}\\





\end{document}
